\begin{document}
  \chapter{Explanation on Routines : Procedures and Functions}
    \section{Procedure 1 : assignroomtobooking}
    \ttfamily
      \begin{lstlisting}[language=SQL]

      DELIMITER $$
      CREATE DEFINER=`root`@`localhost` PROCEDURE `assignroomtobooking`(IN `datein` DATE, IN `dateout` DATE, IN `type` INT, OUT `roomid` INT)
      BEGIN
        SELECT room.idroom as roomid FROM room WHERE (room.idroom
          NOT IN (
            SELECT booking.room_idroom FROM booking
            WHERE (booking.fromdate < datein AND booking.todate > datein)
              OR (booking.fromdate between datein AND dateout - INTERVAL 1 DAY))
                AND room.status = 'open' AND room.roomtype_idroomtype = type) LIMIT 1;
      END$$
      DELIMITER ;
      \end{lstlisting}
      \rmfamily
      \paragraph{}
        This procedure is used in booking function where after pressing confirm booking, the procedure will look for an available room and will output the roomid and will
         be used when insert into the booking table.

    \section{Procedure 2 : availableroomtype}
      \ttfamily
      \begin{lstlisting}[language=SQL]
      DELIMITER $$
        CREATE DEFINER=`root`@`localhost` PROCEDURE `availableroomtype`(IN `infromdate` DATE, IN `intodate` DATE)
        BEGIN
          SELECT *,COUNT(*) FROM room LEFT JOIN roomtype ON room.roomtype_idroomtype = roomtype.idroomtype
          WHERE (room.idroom NOT IN (
            SELECT booking.room_idroom FROM booking WHERE (booking.fromdate < infromdate AND booking.todate > infromdate) OR (booking.fromdate between infromdate AND intodate - INTERVAL 1 DAY)) AND room.status = 'open')
            GROUP BY roomtype.idroomtype;
        END$$
      DELIMITER ;

      \end{lstlisting}
      \rmfamily
      \paragraph{}
        This procedure return the roomtypes that are available for the customer in the time period which he/she selects.

    \section{Procedure 3 : createonlinebooking}
      \ttfamily
      \begin{lstlisting}[language=SQL]
      DELIMITER $$
      CREATE DEFINER=`root`@`localhost` PROCEDURE `createonlinebooking`
      (IN `datein` DATE, IN `dateout` DATE, IN `userid` INT, IN `roomid` INT, IN `coupon` VARCHAR(50), IN `paymentid` INT, OUT `bookingid` INT)
        BEGIN
          	DECLARE couponid INT;
          	SELECT specialoffer.idspecialoffer as couponid FROM specialoffer WHERE specialoffer.code = coupon;
          	INSERT INTO `booking` (`idbooking`, `user_iduser`, `room_idroom`, `specialoffer_idspecialoffer`, `staff_idstaff`, `payment_idpayment`, `fromdate`, `todate`, `status`)
            VALUES (NULL, userid, roomid, couponid, NULL, paymentid, datein, dateout, 'unchecked');
              SET bookingid = (SELECT LAST_INSERT_ID());
        END$$
      DELIMITER ;
      \end{lstlisting}
      \rmfamily
      \paragraph{}
        This procedure facilitate storing in the booking table for online booking.

    \section{Procedure 4 : createonlinepayment}
    \ttfamily
      \begin{lstlisting}[language=SQL]
      DELIMITER $$
      CREATE DEFINER=`root`@`localhost` PROCEDURE `createonlinepayment`(IN `amount` DECIMAL(10,2), IN `uid` INT, IN `cardno` VARCHAR(50), OUT `paymentid` INT)
      BEGIN
        	INSERT INTO `payment` (`idpayment`, `timestamp`, `amount`, `user_iduser`, `method`, `cardno`, `staff_idstaff`,`type`)
          VALUES (NULL, NOW(), amount, uid, 'creditcard', cardno, NULL,'booking');
            SET paymentid = (SELECT LAST_INSERT_ID());
        END$$
      DELIMITER ;
      \end{lstlisting}
      \rmfamily
      \paragraph{}
        This procedure store payment table for online booking.

    \section{Procedure 5 : newwalkin}
    \ttfamily
      \begin{lstlisting}[language=SQL]
      DELIMITER $$
      CREATE DEFINER=`root`@`localhost` PROCEDURE `newwalkin`(IN `inamount` DECIMAL(10,2), IN `instaffid` INT, IN `incardno` VARCHAR(50), IN `inmethod` VARCHAR(20), OUT `paymentid` INT)
      BEGIN

        	INSERT INTO payment VALUES (NULL, NOW(), inamount, NULL, inmethod, incardno, instaffid, 'booking','');
            SET paymentid = (SELECT LAST_INSERT_ID());
        END$$
      DELIMITER ;
      \end{lstlisting}
      \rmfamily
      \paragraph{}
        This procedure store payment for walkin booking and is used in the website for creating booking for walkin customers.

    \section{Procedure 6 : newwalkinbooking}
    \ttfamily
      \begin{lstlisting}[language=SQL]
      DELIMITER $$
      CREATE DEFINER=`root`@`localhost` PROCEDURE `newwalkinbooking`(IN `infromdate` DATE, IN `intodate` DATE, IN `instaffid` INT, IN `inroomid` INT, IN `offer` VARCHAR(50), IN `paymentid` INT, OUT `bookingid` INT)
      BEGIN
  	     DECLARE codeid INT;
  	     SELECT specialoffer.idspecialoffer AS codeid FROM specialoffer WHERE specialoffer.code = offer;
     	   INSERT INTO `booking` (`idbooking`, `user_iduser`, `room_idroom`, `specialoffer_idspecialoffer`, `staff_idstaff`, `payment_idpayment`, `fromdate`, `todate`, `status`) VALUES (NULL, NULL, inroomid, codeid, instaffid, paymentid, infromdate, intodate, 'unchecked');
         SET bookingid = (SELECT LAST_INSERT_ID());
     END$$
     DELIMITER ;
      \end{lstlisting}
      \rmfamily
      \paragraph{}
        This procedure store booking for walkin customer  and is used in the website for creating booking for walkin customers.

    \section{Function 1 : calc\_coupon}
    \ttfamily
      \begin{lstlisting}[language=SQL]
      DELIMITER $$
      CREATE DEFINER=`root`@`localhost` FUNCTION `calc_coupon`(`amount` DECIMAL(10,2), `couponcode` VARCHAR(50)) RETURNS decimal(10,2) unsigned
      BEGIN
        DECLARE dsc DECIMAL(10,2);
        SELECT discount into dsc FROM specialoffer WHERE specialoffer.code = couponcode;
        IF (dsc = NULL) THEN
          return NULL;
        ELSE
          SET dsc = amount*(100-dsc)/100;
            return dsc;
        END IF;
      END$$
      DELIMITER ;
      \end{lstlisting}
      \rmfamily
      \paragraph{}
        This function check whether the coupon in valid or not and return discounted amount.


\end{document}
